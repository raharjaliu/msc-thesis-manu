\documentclass[pdftex,12pt,a4paper]{report}

\usepackage[pdftex]{graphicx}
\usepackage[ansinew]{inputenc}
\usepackage{geometry}
\usepackage{bbold}
\geometry{a4paper,left=2.5cm,right=2.5cm, top=2.5cm, bottom=3cm}
\newcommand{\HRule}{\rule{\linewidth}{0.5mm}}

\begin{document}
\begin{titlepage}


%%LR
\sffamily

\begin{center}


% Oberer Teil der Titelseite:
\includegraphics[width=0.3\textwidth]{logo2.jpg}
\hfill
\includegraphics[width=0.4\textwidth]{logo1.jpg}  
\\[5cm]

{\Large Department of Mathematics}\\[0.5cm]
{\Large Chair of Mathematical Modeling of Biological Systems}\\[0.5cm]
{Technische Universit\"at M\"unchen}\\[2cm]
{\Large Master's Thesis in Bioinformatics}\\[1.5cm]

% Title
\HRule \\[0.4cm]
{ \huge \bfseries Single-cell analysis of cancer drug response using computer vision and learning algorithms on time-lapse microtrench data}\\[0.4cm]

\HRule \\[1.5cm]

{\Large Pandu Raharja}\\[2.5cm]

\vfill
\end{center}
\end{titlepage}
\pagestyle{empty}

%%LR comprehensive title
\begin{titlepage}
{\sffamily


\begin{center}
\includegraphics[width=0.3\textwidth]{logo2.jpg}
\hfill
\includegraphics[width=0.4\textwidth]{logo1.jpg}  
\\[1.5cm]  

{\Large Department of Mathematics}\\[0.5cm]
{\Large Chair of Mathematical Modeling of Biological Systems}\\[0.5cm]
{Technische Universit\"at M\"unchen}\\[1cm]

{\Large Master's Thesis in Bioinformatics}\\[2cm]
{\textbf{\Large Single-cell analysis of cancer drug response using computer vision and learning algorithms on time-lapse microtrench data}}\\[2cm]
{\textbf{\Large Wirkungsanalyse von Krebsmedikamenten in Einzeller Aufl\"osung durch die Anwendung von Computer-Vision- und Machine-Learning-Algorithmen auf Microtrench- Videoaufnahme}}\\[4cm]

\end{center}
\begin{center}\Large
  \begin{tabular}{ll}
    Author:& Pandu Raharja\\
    Supervisor: &  Prof. Dr. Fabian Theis, Dr. Carsten Marr\\
    Advisor:        &  Prof. Dr. Fabian Theis\\
    & Prof. Dr. Dmitrij Frishman\\
    Submitted:     &  15.10.2017
  \end{tabular}
\end{center}

}% end title page

\end{titlepage}


%%%%%%%%%%%%%%%%%%%%%%%%%%%%%%%%%
% thesis content starts here
%%%%%%%%%%%%%%%%%%%%%%%%%%%%%%%%%

\newpage


\begin{abstract}
Quantitative measurement of cancer drug response is esential to objectively gauge the efficacy of cancer drugs. So far, there has been no method to track and  quantitatively measure single-cell response of of cancer drug treatment. A novel pipeline is presented in this thesis. First, a high-throughput method to track cells and quantitatively analyze single-cell response to drugs. We investigate the response of model cancer cell lineagues, MOLM and Jurkat, to known anti-cancer drugs Vincristine and Doxorubicine. Second, a machine learning-based was developed which was able to predict cancer cells' time-to-death upon the introduction of the medications, based on the differential cellular morphology of the cells alone. While these two-fold methods enabled accurate and high-throughput analysis of cancer treatment \textit{in vitro}, our pipeline could also be adapted in varios contexts involving single-cell analysis.
\end{abstract}

\newpage



\chapter{Intro}

Lorem ipsum

\chapter{Background}



\end{document}